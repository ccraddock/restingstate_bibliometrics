%%%%%%%%%%%%%%%%%%%%%%%%%%%%%%%%%%%%%%%%%%%%%%%%%%%%%%%%%%%%%%%%%%%%%%%%%%%%%%%
% Neuroimage-like layout
\documentclass[12pt,3p]{elsarticle}
%\documentclass[5p]{elsarticle}
% For kindle
%\documentclass[1p,12pt]{elsarticle}
%\usepackage{geometry}
%\geometry{a6paper,hmargin={.2cm,.2cm},vmargin={1cm,1cm}}
% End kindle
\usepackage{amsmath,amsfonts,amssymb}
\usepackage{bm}
\usepackage{algorithm}
\usepackage{algorithmic}
\usepackage{url}
\usepackage[breaklinks=true,letterpaper=true,colorlinks,bookmarks=false]{hyperref}
\usepackage[table]{xcolor}

\definecolor{deep_blue}{rgb}{0,.2,.5}
\definecolor{dark_blue}{rgb}{0,.15,.5}

\hypersetup{pdftex,  % needed for pdflatex
  breaklinks=true,  % so long urls are correctly broken across lines
  colorlinks=true,
  linkcolor=dark_blue,
  citecolor=deep_blue,
}

% Float parameters, for more full pages.
\renewcommand{\topfraction}{0.9}        % max fraction of floats at top
\renewcommand{\bottomfraction}{0.8}     % max fraction of floats at bottom
\renewcommand{\textfraction}{0.07}      % allow minimal text w. figs
%   Parameters for FLOAT pages (not text pages):
\renewcommand{\floatpagefraction}{0.6}  % require fuller float pages
%    % N.B.: floatpagefraction MUST be less than topfraction !!


\def\B#1{\mathbf{#1}}
%\def\B#1{\bm{#1}}
\def\trans{^\mathsf{T}}
% A compact fraction
\def\slantfrac#1#2{\kern.1em^{#1}\kern-.1em/\kern-.1em_{#2}}

%%%%%%%%%%%%%%%%%%%%%%%%%%%%%%%%%%%%%%%%%%%%%%%%%%%%%%%%%%%%%%%%%%%%%%%%%%%%%%%%
\begin{document}

\title{Bibliometric Analysis of Resting State fMRI Literature}

\author[cmi]{Matthew K. Doherty}
\author[cmi]{Ayesha Anwar}
\author[cmi]{Sam Barberie}
\author[cmi]{Caitlin Hinz}
\author[cmi]{Michelle Kaplan}
\author[cmi]{Anna Rachlin}
\author[cmi,nki]{Michael P. Milham}
\author[vtcri,vtbme]{Stephen M. LaConte}
\author[cmi,nki]{R. Cameron Craddock\corref{corresponding}}

\cortext[corresponding]{Corresponding author: cameron.craddock@childmind.org }

\address[cmi]{Child Mind Institute, New York, New York}
\address[nki]{Nathan Kline Institute for Psychiatric Research, Orangeburg, New York}
\address[vtcri]{Virginia Tech Carilion Research Institute, Roanoke, Virginia}
\address[vtbme]{School of Biomedical Engineering and Sciences, Virginia Tech,
Blacksburg, Virginia}

\begin{abstract}
    Some cool stuff about why the Child Mind Librarian is so freaking cool ...
\end{abstract}

\begin{keyword}
    Functional connectivity, connectome, group study, effective
    connectivity, fMRI, resting-state
\end{keyword}

\maketitle
%%%%%%%%%%%%%%%%%%%%%%%%%%%%%%%%%%%%%%%%%%%%%%%%%%%%%%%%%%%%%%%%%%%%%%%%%%%%%%%%


\sloppy % Fed up with messed-up line breaks
{
%\clearpage
\section*{References} \small \bibliographystyle{elsarticle-num-names}
\bibliography{biblio} }

%%%%%%%%%%%%%%%%%%%%%%%%%%%%%%%%%%%%%%%%%%%%%%%%%%%%%%%%%%%%%%%%%%%%%%%%%%%%%%%

\section{Introduction}

Since its initial observation in 1995 {Biswal 1995}, resting state functional
magnetic resonance imaging (R-fMRI) has exploded in popularity as a technique
for studying the brain’s functional architecture. Although initially plagued by
	controversies related to the physiological underpinnings and
	unconstrained nature of R-fMRI {Lund 2001}{Morcom 2007}, it has
	persevered to become a standard assessment in basic, clinical and
	cognitive neurosciences {Fox 2010}{Kelly 2012}.  The result is a
	burgeoning literature dedicated to R-fMRI that is quickly becoming
	unwieldy for researchers to navigate. The Child Mind Institute (CMI)
	Librarian initiative addresses this challenge by providing a
	hand-curated reference library of R-fMRI literature
	(http://www.mendeley.com/profiles/cmi-librarian). This library consists
	of 1,721 publications (as of December 24, 2012) and requisite metadata
	to facilitate the systematic review of its contents. We inaugurate the
	availability of this open resource by performing a bibliometric analysis
	to assess the current state of R-fMRI literature.

Bibliometrics involves the application of mathematics and statistical methods to
a body of literature to illuminate its course of development and measure its
impact {Pritchard 1969}. Although primarily concerned with analyzing citations
to estimate the impact of researchers and particular papers within the field,
bibliometrics can also identify common themes within in the literature.
Bibliometric analyses differ from more standard literature reviews in that they
are largely automated and may overlook or misinterpret details of individual
publications that would be apparent to a human reader. On the other hand,
bibliometric analyses are data-driven and can be more comprehensive, using a
larger fraction of the corpus to support its conclusions. The quality of the
analyses is determined by the quality of the literature database used for the
analysis. The CMI Librarian resting state literature database is ideal for
automated data mining because it has been hand-vetted to remove irrelevant
publications, and to provide basic keywords to inform the analysis.

This automated statistical analysis of R-fMRI literature incorporated
information from the CMI librarian database, PubMed (http://pubmed.gov), and the
full text of the publications. Information across these resources was combined
to gain insight into publication growth, venues and patterns. Citation analyses
were performed to identify publications and researchers with the greatest
impact, as well as to identify working groups of researchers who tend to
co-publish. Key word analyses were performed to identify experimental methods,
cognitive domains, clinical disorders, and brain regions most commonly discussed
in the literature. Together the results from these analyses narrate the history
of R-fMRI.

\section{Methods}

\subsection{CMI Librarian}

The CMI Librarian initiative has constructed a comprehensive database of R-fMRI
publications that is maintained in Mendeley (Mendeley, Inc., New York, NY). The
database is updated monthly using PubMed-based searches that are run in Sente
(Third Street Software, Inc., Denver, CO) with the following query: \texttt{``resting
state fMRI'' OR ``intrinsic functional connectivity'' OR ``rest AND functional
connectivity'' OR ``fmri AND default mode network''}. Abstracts corresponding to
publications returned from the search are hand vetted by CMI Librarian staff to
exclude papers that are duplicates, do not have an English language abstract, or
are not explicitly related to fMRI and resting state. Based on the content of
the abstracts, the articles are then assigned tags based on their topic areas.
Articles are marked as \emph{clinical} if they deal with a clinical population of any
kind, and are additionally tagged with the specific disorder/population. The
\emph{basic neuroscience} tag is applied to papers that focus on non-clinical
populations and are additionally categorized as \emph{brain and behavior},
\emph{functional anatomy}, or \emph{multimodal}. The latter refers to papers that integrate fMRI with an
additional imaging modality. There are additional tags for papers that
incorporate genetics (\emph{genetics}) and those that include animal research
(\emph{animal models}). Journal articles that are mainly focused on a particular analytical or
imaging technique are classified with the \emph{methodology} tag. Finally there are
tags for reviews and meta-analyses (\emph{review/meta-analysis}) and papers that use
data from the 1000 Functional Connectomes and International Neuroimaging
Data-sharing Initiative (\emph{1000 Functional Connectomes}). 

\subsection{Publication Trends}
Publication rates, venues, subject areas, and open access policies were measured
and analyzed using metadata found in the CMI Library and complementary PubMed
queries. The growth rates of literature volume over time were found for
R-fMRI and compared to that for all of fMRI - , which is the mother
discipline of R-fMRI. First, the CMI Librarian was aggregated by year to
find each year’s publication count. Then, the following query was used on PubMed
to find the number of articles in all of fMRI per year: \texttt{(``fMRI'' or
``functional
magnetic resonance imaging'') and (``YEAR''[Publication Date])}.  The cumulative
sums were then calculated to find the total literature volume over time.

Growth rates were modeled with exponential functions. An exponential growth rate
indicates that the literature’s growth rate $\slantfrac{dV}{dt}$ is proportional to its
volume ($V$), $\slantfrac{dV}{dt} \propto V$.  To model publication growth, piecewise exponential
functions of the form $V(t)=V_0 e^{Rt}$ were fitted over the intervals
1/1/1994 - 12/31/2005 and 1/1/2006 - 12/31/2012 for both
R-fMRI and fMRI. The year 1994 was chosen as the first point because
growth is subexponential before this period. Between 1994 and 2012, growth
is super-exponential if fitted over the full period; so two piecewise curves
were fitted. The break point at 2006 was chosen by minimizing model error over
the full period. 

Several additional publication trend statistics were computed from CMI Librarian
and PubMed metadata. Counts of publications by journal were found directly from
CMI Librarian data. Clinical applications were aggregated from the library’s
hand-curated tag information. Open access rates for R-fMRIR-fMRI and all of
fMRI were found from PubMed. 

Using address information from the CMI Librarian, a density-equalizing map, or
cartogram, was generated on the publications’ correspondence addresses with the
ScapeToad implementation of the Gastner/Newman diffusion-based algorithm. [11]
The goal of the cartogram, and the Gastner/Newman algorithm in particular, is to
plot a geographical map such that the density of the publications by country is
constant per unit area, and the boundaries of each country are still
recognizable.

\subsection{Term Frequency Analysis}
A bag of words model was used to learn about the field’s experimental methods
and areas of focus. A bag of words model assumes that all of the words in a
corpus are statistically independent, regardless of whether they appear in the
same sentence or publication.  Under this model, a term’s significance can be
measured simply by counting its occurrences.

Terms of interest were derived from n-grams (a phrase consisting of $n$ words)
from neuroimaging methods, cognitive ontology [3, 5], and the PubBrain lexicon
[6]. N-grams from cognitive ontology and the PubBrain lexicon were found from
their respective online sources. A domain expert (RCC) manually generated the
list of terms for neuroimaging methods. The final sets of n-grams were found by
iteratively testing a candidate set, then combining synonyms and compound terms
to form a new candidate set.

For each term, conditional term frequency (conditional-tf) was computed as the
median ratio of term count to total word count in each publication that
contained the term of interest. Document frequency (df) for each term was
computed as the number of documents containing the term. Terms with high df are
popular across the corpus, while terms with high conditional-tf occur often in
the documents in which they appear.  

\subsection{Prevalence Of Methods} 
A Na\"ive Bayes model was used to determine the prevalence of various analysis methods in
the literature. In particular, this model was used to discriminate between
seed-based correlation, ICA, clustering, graph theory, and machine learning
methods. First, a small training set of 10 publications was manually created.
From this training set, important features were derived by finding tokens and
n-grams with high values of the $\chi^2$ statistic, which is used as a proxy
for information gain. (Quinlan, Induction of Decision Trees.) A term frequency
	matrix was then constructed, relating each publication to the frequency
	of each term contained in the publication. Logarithmic term frequencies
	($\log{t\!f}$) were used to prevent a single term from dominating the model.
	Each publication’s vector was normalized to unit length to mitigate the
	effect of publications that contain too many or too few terms. This term
	frequency matrix was used to train a Na\"ive Bayes model. Finally, 213
	additional publications were classified by hand and used to test the
	classification accuracy of the model.

\subsection{Citation Analysis}
Publications and the citations that link them were modeled as a directed graph
to characterize their relationships and ``small world'' nature [10]. A directed
graph is a general framework used to represent relationships between nodes
(publications) using edges (citations), which each point from one node to
another.

First, PDF files corresponding to the titles in the CMI Librarian were
systematically downloaded. A fuzzy search for every publication title in every
file was performed using the SequenceMatcher method of Pythons difflib library
(http://docs.python.org/2/library/difflib.html). This fuzzy search first cleaned
the binary file with regular expressions to remove junk patterns and normalize
whitespace. The longest identical match of the publication title was compared
to this match and its context to the true publication title by finding the match
ratio (measure of similarity between strings). Of 20,541 total matches with
ratio greater than or equal to 0.8, 16,425 (80\%) had a ratio of 1.0—i.e., they
were exact matches—and 17,983 (88\%) had a ratio at least 0.9. The cutoff for
true matches was set to 0.9, resulting in 17,983 citations. 

Using the citations found by the fuzzy search, a graph was constructed of the
CMI Librarian publications (nodes) and the citations between them (edges). From
this graph, the most central nodes were found using the NetworkX pagerank
implementation {Hagberg 2008}. Pagerank, a variant of eigenvector centrality,
aims to find the steady state probability of a random walk reaching each node
{Brin 1998}. These solutions are given by the equation:

\begin{equation}
P\!r(p_i)=\frac{1-d}{N} + d\!\sum_{p_j \in M(p_i)}\frac{P\!r(p_i)}{L(p_j)}
\end{equation}

\noindent where $N$ is number of nodes, $d$ is the
probability of continuing the random walk, $M(p)$ is the set of pages that link to
node $p$, and $L(p)$ is the number of outbound edges from $p$. The solutions,
$P\!r(p_i)$,
are the dominant eigenvector of a modified adjacency matrix, and can be found
with the power method.

Finally, a jackknife procedure {Quenouille 1956} was performed to reduce the
variance of the pageranks and to quantify their sensitivity to choices of
fuzzy-search parameters. For each of over 1,000 replicates, 20\% of the edges were randomly
deleted, and Pagerank was calculated.  The pagerank mean
and standard error over the jackknifes were reported,which is an overestimate of the true procedure error.  
Additionally, the graph's mean shortest path length and mean clustering coefficient was
calculated in a seperate analysis.


\subsection{Graph Analysis of Co-authorships}
Collaborative relationships between authors of R-fMRI literature can be
characterized by co-authorships. A graph was constructed from the CMI Librarian data in which each
node corresponds to an author and edges represent co-authorships, weighted by
the number of papers that the two authors appear on. From this graph, we
calculated average path length, clustering coefficient, Pagerank centrality,
and the quantity of disjoint sets. Additionally, we tested for graph robustness
by repeatedly removing authors and publications and measuring graph
connectivity.

\subsection{Working Groups}
We defined working groups as sets of researchers who frequently publish together
and, once found, the publication patterns of these groups were examined. In
order to find these groups, a greedy community-detection algorithm was developed
that uses the intersection of researchers’ publication sets to measure their
similarity.

The procedure for identifying author working groups begins by repeatedly seeding
a new working group with the author with the most publications (lines 2-3).
Authors are searched to find the author with the most co-authored publications
with the working group (line 6). If this author has more than 10 publications,
and is a co-author on at least 30\% of the publications authored by the members
of the working group (line 7), then the author is added to the working group
(line 6). Otherwise, the working group is closed and considered for inclusion in
the results. If the sum of all papers in common across the candidate working
group is greater than 50, and the working group has less than 20% of authors in
common with any other working group (line 8), then it is included in the results
(line 9).

%% Algorithm goes here!!

\section{Results}

\subsection{Publication Trends}
Growth of R-fMRI literature was fitted by piecewise exponential functions with
32\% growth between January 1,1994 and December 31, 2005 and 47\% between January
1, 2006 and December 31, 2012. In comparison, growth of fMRI literature was
found to be 26\% and 17\% during the same time periods. Figure 1 shows the growth
of each literature, and represents a total of 28,434 fMRI publications and 1,721
R-fMRI publications. Additionally, 39\% of R-fMRI publications were determined to
be open access, compared to 31\% in all of fMRI.

The top 20 publication outlets for R-fMRI publications accounted for 59\% (1017)
of the CMI R-fMRI library as shown in Fig. 2. This list is dominated by three
neuroimaging journals (NeuroImage, Human Brain Mapping, Brain Connectivity),
which accounted for approximately 26\% of the R-fMRI publications. The top 20
also include nine general neuroscience, two general science, four clinical
neuroscience, and two general imaging journals which accounted for approximately
16\%, 10\%, 5\%, and 3\% of the library respectively. The remaining 41\% of the
library was spread over 268 different publication outlets. Nine of the top 20
journals began publishing R-fMRI literature before 2006, and Brain Connectivity,
founded in 2011, is the most recent addition.  

The density-equalizing map is shown in figure 3. The top three countries are the
United States of America, United Kingdom, and Netherlands, wherein lie 62.9\%,
9.6\%, and 8.6\% of correspondence addresses, respectively.


\subsection{Term Frequency Analysis}

\begin{table}[H]
\caption{\label{tagfreqtable} The number and fraction of publications in the
library tagged with a specific keyword}
\begin{center}
\begin{tabular}{|l|r|}
\hline
{\bf CMI Librarian Tag}&{\bf Frequency} \\ \hline \hline
Clinical & 747 (43\%) \\ \hline
Basic Neuroscience&593 (34\%) \\ \hline
Meta-analysis/reviews&235 (14\%) \\ \hline
Methodolog&210 (12\%) \\ \hline
Multimodal&143 (8\%) \\ \hline
Brain and Behavior&79 (5\%) \\ \hline
1000 Functional Connectomes&61 (4\%) \\ \hline
Animal Models&61 (4\%) \\ \hline
Functional Anatomy&44 (3\%) \\ \hline
Genetics&29 (2\%) \\ \hline
\end{tabular}
\end{center}
\end{table}


Growth of the most prevalent clinical terms is shown in figure 3. Of 1,721
publications in the corpus, ``Clinical'' is the most commonly applied tag
(Tab. \ref{tagfreqtable}). ``Schizophrenia'' (13\%), ``Alzheimer's Disease'' (11.3\%),
and ``Depression'' (10.8\%) are the most common clinical sub-tags that co-occur with the
``clinical'' tag.

Figure 5 shows growth of the term ``connectome'' in the corpus. Term frequency
($t\!f$), document frequency ($d\!f$), and term frequency-inverse document frequency
($t\!f \times d\!f^{-1}$) have all increased since 2009.  

Figure 6 shows conditional-$t\!f$ and $d\!f$ for the terms with the highest values. The
most common imaging modality was fMRI, which was a key term in the searches used
to generate the corpus. The most investigated cognitive domains were activation,
memory, attention, and association. The PFC, PCC, and anterior cingulate were
the most discussed brain regions. 

\subsection{Methods Analysis}

The accuracy of the Na\"ive Bayes model in inferring
algorithmic methods is shown in \ref{??}.

The terms used as features, along with their $\chi^2$ values, are shown in
\ref{??}.

Figure 6 shows growth over time of algorithmic methods in R-fMRI literature.
Seed-based correlation is by far the most common method, accounting for nearly
half of all uses of algorithmic methods. Independent component analysis (ICA)
and clustering have shown steady growth since 2010, and use of machine learning
methods increased at a faster pace in 2012.

X\subsection{Citation Analysis} 
The top 10 publications by pagerank were collectively cited by 66\% of the
corpus, and the top 1\% of publications account for 10\% of the total pagerank.
After these publications, pagerank falls off more slowly, with the next 10\% of
publications accounting for 40\% of the total pagerank. 

The mean clustering coefficient was 0.094 (standard error 0.010) and the mean
shortest path length was 4.4 (standard error 0.080). In a set of 1,000 random
graphs constructed on the same set of vertices, the mean clustering coefficient
was 0.014 (standard error 0.0030) and the mean shortest path length was 4.1
(standard error (0.60).

\subsection{Co-authorship Analysis}
The co-authorship graph was found to be robust, exhibiting significant
small-world statistics. In particular, its clustering coefficient was 0.878,
compared to 0.002 for a random graph with the same number of nodes and edges,
and its average shortest path length was 4.964, compared to 3.915 for the random
graph.

Figure 11 shows the number of connected components, and size of the largest
connected component, as publications and authors are removed.  

\subsection{Working Groups Analysis}
Four working groups were identified. Their
combined 23 authors (0.6\% of 3,704 across the literature) cover 17.5\% of R-fMRI
publications. The mean publication count by authors in these groups was 21.9;
overall, it was 0.31. The mean number of publications in common between pairs of
authors in the same working group was 10.7; across different working groups, it
was 0.11.  

Working group authors (number of publications in corpus by author):
\begin{enumerate}
\item Michael P. Milham(41), Clare Kelly(36), F. Xavier Castellanos(37), and
Bharat Biswal(36) 
\item Tianzi Jiang(35), Kun-Cheng Li(27), Chunshui Yu(19),
Li-Xia Tian(18), Yuan Zhou(15), and Yong Liu(15) 
\item Qi-Yong Gong(35), Yi-Jun Liu(21), Wei Liao(20), Hua-fu Chen(20),
Guang-Ming Lu(18), Zhiqiang Zhang(17), and Yuan Zhong (13) 
\item Bradley L. Schlaggar(19), Steven E. Petersen(16), Alexander L.
Cohen(13), Damien Fair(13), Nico U. F. Dosenbach(10) and Fran M. Miezin(10)
\end{enumerate}

\section{Discussion}

Our bibliometric analysis of R-fMRI literature lends insight into the current
state of the field, demonstrating its strength, areas of focus, and future
potential. It was found that the growth of R-fMRI literature is currently
significantly faster than fMRI, though R-fMRI currently comprises a small
fraction of the total volume of fMRI literature. In academic research, a high
rate of publication growth indicates not only that researchers in the field are
publishing more, but also that additional researchers are publishing in the
field. The journal Neuroimage has published more R-fMRI literature than any
other. In general, nearly 50\% of the literature was published in general
neuroimaging journals, perhaps reflecting the rapid advancement of methods used
to study R-fMRI over the past 10 years. 

In contrast, only 1\% was published in clinical journals, suggesting that the
field has not yet matured enough to be used in clinical applications. But
despite their venues, a full 35\% of the R-fMRI corpus was tagged “Clinical.”
Indeed, a great proportion of R-fMRI research is performed on both clinical and
control populations in order to identify differences between them; typically,
however, the results of these experiments are not immediately applicable in
clinical medicine.

Another pattern that was explored, open access, is a relatively recent
innovation that allows researchers to access publications without payment to the
journal by either front-loading the cost onto the author or subsidizing the cost
in another way. Open access is not universal in R-fMRI (26\%) or fMRI (22\%), but
has strong footholds in both. That the rate of open access is consistently
increasing over time in both literatures bodes well for the future of open
science.

Word frequency analysis easily identified the corpus domain, fMRI, as the most
common imaging modality. In addition, there was a focus on the prefrontal cortex
(PFC), which is implicated in executive function, as well as on the posterior
cingulate cortex (PCC), which is a central node of the DMN used for integration.
Notably, neither the parietal cortex nor any of its subregions appear in the top
hits, though some are implicated in the DMN. This result may demonstrate the bag
of words model’s limited ability to aggregate concepts: for example, it may be
the case that the sum of occurrences of subregions of the parietal cortex would
appear in the top hits, but none appear individually. Finally, memory and
activation were the most discussed cognitive domains, reflecting current
research trends such as in [8] and [9]. A possible extension of this research is
to identify common terms across scientific fields to determine, for example,
whether particular genes can affect PFC and PCC function.

Analysis of working groups found particularly prolific labs and collaborators,
offers measures for whether the field is dominated by only a few authors, and
found patterns in geographic localities and language. The resulting groups are
together responsible for nearly a fifth of the total corpus, including some of
the highest-pageranked publications, though there is little publication overlap
between groups. Ranking authors by their impact factor reveals that several of
the authors in these groups, including M P Milham, F X Castellanos, A M Kelly, B
Biswal, and Qi-Yong Gong, are not only prolific, but also are attributed many
citations. Furthermore, geographic location, language, and affiliation are
consistent within groups, but very different between groups: M P Milham’s group
is based in New York, T Z Jiang’s group is based in Beijing, Qi-Yong Gong’s
group is based in Sichuan, and Bradley Schlaggar’s group is based in St. Louis.
This result demonstrates the strength of R-fMRI research around the world, and
suggests that distance, time zone, and language pose practical barriers to
collaboration.

Analysis of the citation graph concluded that it can be considered ``small
world'' because its mean clustering coefficient is significantly higher than
that of a random graph, and its mean shortest path length is not significantly
different from that of a random graph. [10] In a small world network, most nodes
are not adjacent to one another, but the mean path length is still fairly short
because there are few degrees of separation needed to reach one node from
another. More precisely, the mean path length of a small world network grows
with the logarithm of the total number of nodes. In this application, this
result indicates that most publications do not cite a large fraction of the
corpus, but the number of hops needed to traverse from one publication to
another is still fairly short. Indeed, there are many real-world relationships,
such as social networks and Internet connections, that can be represented with
small world graphs. 

By measuring pagerank, publications were identified that are central to R-fMRI
literature without necessarily having the highest raw citation count. The
highest pageranked publications were found to be seminal publications that
provide much of the historical groundwork for the field. Indeed, from these
publication emerges a narrative of R-fMRI history, from the foundations of
R-fMRI functional connectivity and activation, to the first clinical application
in the field, to the discovery of the canonical independent R-fMRI networks.
Relatively consistent pageranks among the remaining publications suggest that
the field is not completely represented in only a small subset of the corpus,
but rather is still growing in high-impact directions. 

The publication with the second highest pagerank, Biswal et al’s ``Functional
connectivity in the motor cortex of resting human brain using echo-planar MRI''
(1995), demonstrated that functional connectivity—spontaneous, low frequency
activity correlated between functional regions—is present not only during task,
but also during R-fMRI. Resting state functional connectivity had been observed
in clinical settings for a long time using EEG, which measures frequency bands,
and PET, which measures glucose metabolism. However, this was the first time it
was observed in the complex FMRI-measured BOLD signal, which lacks clear
physiological underpinning. Biswal appears in a top working group and in the
list of highest impact authors. Functional connectivity, first demonstrated with
R-fMRI in this publication, is among the most frequently occurring terms in the
corpus. 

Another, separate line of research took root with a series of 2001 publications
that include those with first and seventh highest pagerank, Raichle et al’s ``A
default mode of brain function,'' and Gusnard et al’s ``Searching for a baseline:
functional imaging and the resting human brain.'' These publications established
that decreases of brain activity that occur during rest are spatially
consistent, and introduced the notion of a functional default mode. This line of
research on activation was cognitive neuroscience’s contribution to the
foundations of R-fMRI theory. Indeed, activation was found to be among the most
frequent cognitive terms in the corpus.

These disparate contributions from neuroimaging and cognitive science were
married in the third highest pageranked publication, Grecius et al’s ``Functional
connectivity in the resting brain: a network analysis of the default mode
hypothesis'' (2003). Grecius showed that, in fact, the regions that show
correlation during rest are also activated during rest. Thus, this publication
introduced the notion of the default mode network, whose regions show both low
frequency correlation of spontaneous activity during rest, and also decreased
task-related activity. Notably, the default mode network is among the most
common neuroimaging terms in the corpus.

In 2004, Grecius followed with the fifth highest pageranked publication,
''Default-mode network activity distinguishes Alzheimer’s disease from healthy
aging: evidence from functional MRI.'' This work linked clinical neuroscience
with R-fMRI functional connectivity by demonstrating with fMRI not only that
resting state activity was reduced in Alzheimer’s patients, but also that the
regions in which R-fMRI activity was reduced were the very same that had been
previously implicated in Alzheimer’s disease. 

In 2006, DeLuca et al’s ``fMRI resting state networks define distinct modes of
long-distance interactions in the human brain'' found resting state
networks—regions that are individually correlated during rest—that are
independent both functionally and statistically. These networks have been
applied widely since their discovery, and appear in the list of most common
neuroimaging terms in the corpus.


\section{Conclusion}

Our bibliometric analysis demonstrates that, though still small compared to
fMRI, R-fMRI is a rapidly-growing field with major international research hubs.
The research community is tight-knit, as shown by its small world citation
network, and the field is no one-trick pony, as shown by the large number of
highly-pageranked publications. The word frequency analysis identified key
concepts in use, and suggests future paths toward integration of R-fMRI with
other rapidly-growing fields such as genetics.

\section{Acknowledgements}
This research was supported by the Child Mind Institute Endeavor Scientist program.

\end{document}
